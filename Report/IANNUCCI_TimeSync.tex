% !TEX TS-program = pdflatex
% !TEX encoding = UTF-8 Unicode

% This is a simple template for a LaTeX document using the "article" class.
% See "book", "report", "letter" for other types of document.

\documentclass[12pt]{article} % use larger type; default would be 10pt

\usepackage[utf8]{inputenc} % set input encoding (not needed with XeLaTeX)

%%% Examples of Article customizations
% These packages are optional, depending whether you want the features they provide.
% See the LaTeX Companion or other references for full information.

%%% PAGE DIMENSIONS
\usepackage{geometry} % to change the page dimensions
\geometry{a4paper} % or letterpaper (US) or a5paper or....
% \geometry{margin=2in} % for example, change the margins to 2 inches all round
% \geometry{landscape} % set up the page for landscape
%   read geometry.pdf for detailed page layout information

\usepackage{graphicx} % support the \includegraphics command and options

% \usepackage[parfill]{parskip} % Activate to begin paragraphs with an empty line rather than an indent

%%% PACKAGES
\usepackage{booktabs} % for much better looking tables
\usepackage{array} % for better arrays (eg matrices) in maths
\usepackage{paralist} % very flexible & customisable lists (eg. enumerate/itemize, etc.)
\usepackage{verbatim} % adds environment for commenting out blocks of text & for better verbatim
\usepackage{subfig} % make it possible to include more than one captioned figure/table in a single float
% These packages are all incorporated in the memoir class to one degree or another...

%%% HEADERS & FOOTERS
\usepackage{fancyhdr} % This should be set AFTER setting up the page geometry
\pagestyle{fancy} % options: empty , plain , fancy
\renewcommand{\headrulewidth}{0pt} % customise the layout...
\lhead{}\rhead{Iannucci \thepage}
\lfoot{}\cfoot{}\rfoot{}

% Alter some LaTeX defaults for better treatment of figures:
    % See p.105 of "TeX Unbound" for suggested values.
    % See pp. 199-200 of Lamport's "LaTeX" book for details.
    %   General parameters, for ALL pages:
    \renewcommand{\topfraction}{0.9}    % max fraction of floats at top
    \renewcommand{\bottomfraction}{0.8} % max fraction of floats at bottom
    %   Parameters for TEXT pages (not float pages):
    \setcounter{topnumber}{2}
    \setcounter{bottomnumber}{2}
    \setcounter{totalnumber}{4}     % 2 may work better
    \setcounter{dbltopnumber}{2}    % for 2-column pages
    \renewcommand{\dbltopfraction}{0.9} % fit big float above 2-col. text
    \renewcommand{\textfraction}{0.07}  % allow minimal text w. figs
    %   Parameters for FLOAT pages (not text pages):
    \renewcommand{\floatpagefraction}{0.7}  % require fuller float pages
    % N.B.: floatpagefraction MUST be less than topfraction !!
    \renewcommand{\dblfloatpagefraction}{0.7}   % require fuller float pages

    % remember to use [htp] or [htpb] for placement

%%% SECTION TITLE APPEARANCE
\usepackage{sectsty}
\allsectionsfont{\sffamily\mdseries\upshape} % (See the fntguide.pdf for font help)
% (This matches ConTeXt defaults)

%%% ToC (table of contents) APPEARANCE
\usepackage[nottoc,notlof,notlot]{tocbibind} % Put the bibliography in the ToC
\usepackage[titles,subfigure]{tocloft} % Alter the style of the Table of Contents
\renewcommand{\cftsecfont}{\rmfamily\mdseries\upshape}
\renewcommand{\cftsecpagefont}{\rmfamily\mdseries\upshape} % No bold!

%%% END Article customizations

%%% The "real" document content comes below...

\title{}
\author{}
%\date{} % Activate to display a given date or no date (if empty),
         % otherwise the current date is printed 

\begin{document}
\section{Time Synchronization}
\subsection{Problem}
The BeagleBone Black does not include a Real Time Clock (RTC). This inherently renders the internal clock unusable for reliable timing unless it can synchronize with an outside source. For this adjustment there are two options. The first is to synchronize with a network clock using NTP. This is the solution used by most laptops to supplement their bundled RTC to achieve increased accuracy. Dependant on the source, the link can provide time accuracy to tens of microseconds \cite{NTP}. However, the typical accuracy falls around ten milliseconds \cite{GPSD}. The second option is to remove the middle man and create a time server on each server using a Global Positioning System (GPS) to set the clock. This method can result in a clock with a maximum of 100 milliseconds of error, which is unlikely \cite{GPSD}. For this design, the latter method was chosen based on the accuracy falling within an acceptable range and the lack of an available Wide Area Network (WAN) connection. Using individual GPS devices results in a modular system that allows for compatibility with potential mesh networks. 

\subsection{Synchronizing Using GPS}
While the GPS atomic clock reports reports accuracy to 50 nanoseconds, serial interfaces are much slower with time accuracy to 1 millisecond \cite{GPSD}. Thus, a Pulse Per Second (PPS) signal must be used to adjust and calibrate the time. A PPS signal is a high accuracy oscillator used to increase inter-device connectivity accuracy \cite{PPS}. With correct support, a PPS is typically accurate to 5 milliseconds \cite{GPSD}. This signal is used to adjust and synchronize the system time beyond the capability of serial GPS data. Therefore, several packages using the same PPS capable GPS receiver can be adjusted to have the same time to accuracy of less than a millisecond without any inter-package communication.

\subsection{Implementation}
Using a Linux kernel allows for interfacing a PPS signal at a low level. The GPSD program allows for the interfacing NMEA standard GPS devices along with PPS support. On Linux systems, GPSD is capable of combing both NMEA sentences and PPS signals \cite{GPSD}. The BeagleBone Black receives the NMEA serial messages through a UART connection and the PPS signal through a GPIO pin. the Network Time Protocol Daemon for Linux allows for multiple input sources to increase accuracy. The NTPD program is responsible for setting the time on standard Linux operating systems. GSPD has the ability to broadcast the PPS and GPS time data on a local socket to make it available for third party clients. Specifically, NTPD can be set to register the sockets published by GPSD as sources. NTPD can then use the GPS data to set the system clock with accuracy below a millisecond. When multiple packages have this same setup, their clocks will be synchronized to within a millisecond. For this application, that accuracy is acceptable. Once the system clock is set, the software follows the design outlined in the Software Design section. 

\bibliography{IANNUCCI_TimeSync}
\bibliographystyle{plain}
\end{document}