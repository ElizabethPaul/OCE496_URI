\begin{abstract}
Structural health monitoring is the process of continuously
monitoring the vibrations of a structure and analyzing long term
changes in the natural frequency to detect internal or external
damage and its location, type, and severity. By the completion of
this project, multiple sensor packages intended to monitor the
structural health of the Claiborne Pell Newport Bridge will be
designed, built, and installed. In the initial phase of this
project, the bridge was considered to be a simply supported
beam. Sensor packages consisting of an accelerometer and a
strain gauge, with supporting microprocessor and analogue
to digital converter were secured to both an I-beam and
angle beam to measure vibration data. These beams were
chosen as preliminary representations of the bridge as
they will have comparable modal responses. A finite element model
(FEM) was produced to calculate the modal responses of the beams,
which were used to determine the optimal locations for the sensor
packages. The FEM analyses were verified with analytical
solutions and experimental data, confirming the natural
frequencies for the I-beam and the angle beam. The sensor
packages were installed on the angle beam and modes 1-5 were
detected and matched the theoretical analyses and FEM. Sensor
packages similar to this are currently availabe, however, no
packages offer wireless communication capability, power
independence, and a minimal power draw at a reasonable
price. In the second phase of this project the sensor
packages will be modified for installation on the Pell
Newport Bridge.

\end{abstract}