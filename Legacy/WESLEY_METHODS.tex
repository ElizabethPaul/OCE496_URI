

% Data collection------------------------------------------------------------------------------------
\chapter{Data Collection}
\subsection{Camera}
\indent A Contour 1500 version 1.0 was used to record audio video files, in a .MOV format. The I-beam was oriented with the flange facing on the ground, supported by two steel pins. The top flange was struck with a claw hammer at the center of the beam and allowed to resonate until the vibrations had completely dissipated. Next the beam was rotated such that the web was level with the ground. The web was struck with a claw hammer at the center of the beam and allowed to resonate. 

\indent The files were converted to wave (.WAV) audio files for data analysis. The microphone on the camera is assumed to have a flat frequency response. The maximum observable frequency is 2400 Hz, based on Nyquist sampling theorem for a sampling rate of 4800 Hz. MATLAB was used to view a time series of the signal and compute the fast Fourier transform (FFT) of the signals.

Figure \ref{fig:RES_Cam_WebFlange} shows clipping for a very brief period. The signal can be seen decaying very rapidly.

\begin{figure}
\centering
\includegraphics*[width = 0.75\linewidth]{WebFlange.eps}
\caption{\textit{Initial time series data of I-beam excitation test}}
\label{fig:RES_Cam_WebFlange}
\end{figure}

\indent The FFT of the signal, \ref{fig:IbeamFFT} shows some variance between the spectral data of the two signals but both share a common peak at 58.6 Hz proving that regardless of which portion of the beam is struck the entire beam will resonate as a single body rather than different components vibrating independently.  The beam appears to be too rigid during the experiment. A first mode at 58.6 Hz is relatively large compared to the full size bridge with a frequency on the order of 1 Hz. A longer less rigid beam was suggested to lower the frequency. 

\begin{figure}
\centering
\includegraphics*[width = 0.75\linewidth]{Ibeam}
\caption{\textit{FFT for web and flange}}
\label{fig:IbeamFFT}
\end{figure}

\subsection{Piezo Electric Strip}
\indent A DT series piezo electric strip was mounted with 3M double sided foam tape to the inside of the L beam. As the beam flexes a voltage is generated by the sensor and captured as a coma separated value (.CSV) file on a Tektronix oscilloscope. The factory calibration is 10 millivolts per micro strain. 

\indent The cursor tool on the oscilloscope was used to measure the period of one oscillation for a quick measurement of the frequency of the primary mode of the beam. A FFT was performed on a one second clip recorded on the oscilloscope

\begin{figure}[H]
\centering
\includegraphics*[width = 6in]{oscopeTIME.eps}
\caption{\textit{Time-series data for piezo-electric strip testing}}
\label{fig:RES_PEST}
\end{figure}

\indent A peak frequency of 3.05 Hz was recorded using the piezo-electric strip, see Appendix \ref{app:AngleFFT1.jpg}.

\subsection{3g Tri-Axial Accelerometer}
\indent The ADXL330 3g accelerometer was mounted on a breadboard, spring clamped to the center of the beam. The outputs of the three axes were sampled on a National Instruments BNC-2110 DAQ. The beam was deflected and released until no more vibrations were observed. The process was repeated with the accelerometer at 1/3, 1/4, and 1/5 the length of the beam.

\indent Upon initial analysis of the time series, it was observed that the signal did not change regardless of accelerometer placement, Figure \ref{fig:onemode}. All four experiments show the beam vibrating at 2.68 Hz. 

\indent The results from the prior experimental method proved unsuccessful and resulted in changing the method of exciting the beam. In order to excite higher order modes a hammer was used to strike the bean at 1/8, 1/10, 1/16 the length of the beam for the four accelerometer locations. The full time series plot for the 3 strike locations and 4 accelerometer locations can be seen in Appendix \ref{app:RES_3g_T_ALL}. 

\begin{figure}[H]
\centering
\includegraphics*[width = 6in]{onemode.eps}
\caption{\textit{Time series for 3g tri-axial accelerometer data}}
\label{fig:onemode}
\end{figure}

The data from the experiment with 4 accelerometer locations and 3 strike locations produced higher quality data than the previously performed experiments. \ref{fig:clipping}, shows that the accelerometer exceeded the 3g threshold, 2.886 volts in several of the trials. 

\begin{figure}[H]
\centering
\includegraphics*[width = 6in]{clipping}
\caption{\textit{Clipped data from fifth 16 iteration}}
\label{fig:clipping}
\end{figure}

A filter was tested on unclipped data to see if there were any higher frequencies that were being aliased. A low pass filter set to 300Hz was selected because it matched the bandwidth of the accelerometer. Figure \ref{fig:filter} shows the filtered data matches with the raw data. 

\begin{figure}[H]
\centering
\includegraphics*[width = 6in]{filter}
\caption{\textit{300 Hz lowpass filtered}}
\label{fig:filter}
\end{figure}

\indent The Y and Z axis are both in the plane orthogonal to the length of the beam. Although the time series shows that the Y and Z axis do not move in phase, the FFT shows that both channels have the same modes. To reduce the amount of computations in further experiments an analysis of only one channel was performed.

\subsection{6g Tri-Axial Accelerometer}
\indent Figure \ref{fig:DC_AccPlacingIdeal} shows the strike location and accelerometer location to capture the first five idealized modes. By leaving the accelerometer at mid span all the odd modes should show while the even modes will be suppressed because the accelerometer is on a node. 

\begin{figure}[H]
\centering
\includegraphics*[width =6in]{ideal5.jpg}
\caption{\textit{Plot showing the ideal location for mounting the accelerometer in order to capture the first five modes of vibration}}
\label{fig:DC_AccPlacingIdeal}
\end{figure}
\indent The time series recorded to the internal memory on the microprocessor was converted into spectral data using an FFT. Due to hardware limitations of 3 input channels at the time of testing the Y and Z channels of the MMA7361L 6g accelerometer were connected to the external ADC on the BBB, as well as the output of the strain gauge. The X channel was ignored because the beam will not accelerate in the direction of the beam because of the boundary conditions. The Wheatstone bridge was not used because differential inputs were not available. The reference voltage was assumed to be constant at half of the input voltage, generated from the bench top power supply. A 120 $\Omega$ resistor was still used to complete the voltage divider circuit. A time series of the three channels was recorded. 