\indent Effort has been made in recent years toward structural health monitoring development as evaluating the fatigue of existing infrastructures will improve will safety. According to the National Bridge Inventory, $90\%$ of all daily traffic utilizes a state or federal owned bridge so the need of ensuring that these structures are properly maintained and rebuilt when necessary is imperative \cite{RobertS.Kirk:2007}. \\
\indent In the 1950s the Interstate Highway System inspired a building surge that resulted in what are now referred to as the baby boomer bridges. Unfortunately, these bridges were built to last for 50 years as it was anticipated that with the development of materials and technology these bridges would be obsolete in that time \cite{AmericanAssociationofStateHighwayandTransportationOfficials:2008}. An unforeseen economic hardship both in the country’s compromised financial situation and in dramatic construction cost increase has made this replacement process near impossible. In recent years immediate closures prompted by large scale structural hazards in the form of cracks or corrosion lead to huge displacements of traffic \cite{AmericanAssociationofStateHighwayandTransportationOfficials:2008}. \\
\indent Structural health monitoring can detect internal damage by sensing the resulting vibration change the bridge experiences. The source of this damage can be attributed to long term fatigue from aging, and short term events such as weather, flooding, or traffic. By interpreting these vibration changes the location and extent of damage can be potentially deciphered. Without visual inspection of the bridge large-scale failure can be evaded, making the bridges safer and avoiding immediate closures. \\
\indent In designing and prototyping a structural health monitoring sensory package for the Caliborne Pell Newport Bridge finite element modeling, confirmed by analytical solution, need to be explored. The foremost concern of the structural health of the Pell Bridge is to ensure structural integrity and guarantee the safety of the travelers. The recent collapse of the I-35W Bridge in Minneapolis and the quarter of bridges in the United Stated that were deemed structurally deficient according at a recent structural survey, has provoked the development of structural health monitoring systems and the need for standards to be set. If a structure is continuously monitored then damage can be detected by the changes reflected in the data. Accurate sensory systems can detect that damage has occurred, located the damage, and evaluate the extent of the damage. With the widespread adaptation of monitoring systems data can be compiled and used to produce standards. 
