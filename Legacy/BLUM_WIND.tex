\subsubsection{Wind Power}
\indent The site of the Newport Bridge is a suitable location for a
wind turbine because the bridge is high off the ground and there
are no sounding objects to block the wind. The process of
harvesting energy from the wind is simple; wind turns a generator
to create electricity. The electricity is then stored in
batteries. In addition to the wind turbine a solar panel will
be installed to supplement power generation. Due to cost and
space limitations, the minimum quantity of batteries required
to keep the sensor package running will be used. The sensor
 package will require a continuous power source in order to
  collect data. The goal is to collect five years of data
   from the sensor package. \\   
   
\indent A NOAA Station  (National Oceanic and Atmospheric
Administration) located about 500 meters north of the bridge in
Newport, RI collects data continuously every 6 minutes,
corresponding to about 87,000 data points a year. The anemometer
(wind speed gauge) is approximately 6.4 meters off the ground.\\    
\indent For the wind speed to be accurate at the height of where the
turbine would be placed on the bridge, the wind data has to be
reconfigured. This is done with the wind profile power law: 

\begin{equation}
U_{2} = U_{1}(\frac{Z_{2}}{Z_{1}})^\alpha
\label{eqn:WIND_Power_Law}
\end{equation}
Where $U_{2}$ the wind is speed at height $Z_{2}$, $U_{1}$ and is
the wind speed at height $Z_{1}$. $Z_{2}$ is the computed height
and is the reference height. is the power law exponent. The power
law exponent is a function of the local climatology, topography,
surface roughness, environmental conditions, meteorological
lapse rate, and weather stability \cite{Zekai}. This makes the
wind profile to be logarithmic and not linear. Studies have
shown that the power law exponent to be .14 or 1/7 for most
sites. The wind data from NOAA, Figure \ref{fig:WIND_6.4},
 was put into MATLAB and the wind data for the corresponding
  height of 50 meters, Figure \ref{fig:WIND_50}, was
   calculated. The average wind speed at the NOAA station is
    $4.38 (m/s)$ and the average speed at the corresponding
     height of 50 meter was $5.83(m/s)$.

\begin{figure}[H]
\centering
\includegraphics*[width = 6in]{WIND_Data_6}
\caption{\textit{Time series for wind data collected in 2012 at 6.4 meters}}
\label{fig:WIND_6.4}
\end{figure}

\begin{figure}[H]
\centering
\includegraphics*[width = 6in]{WIND_Data_50}
\caption{\textit{Time series for wind data collected in 2012 at 10 meters}}
\label{fig:WIND_50}
\end{figure}

\indent The computed wind data was then analyzed and a probability
density graph of the wind speed what created in MATLAB. The
probability density graph, figure 3, shows the probability of each
wind speed.

\begin{figure}[H]
\centering
\includegraphics*[width = 6in]{WIND_PDF}
\caption{\textit{Probability Density Function of computed data}}
\label{fig:WIND_PDF}
\end{figure}

\indent The probability density graph gives a more accurate
calculation of the power output. This is because the equation for
power is:

\begin{equation}
Power = \frac{A_{S} \times A_{D} \times E \times V^{3}}{2}
\end{equation}

\indent Where $A_{S}$ is the swept area of the turbine blade,
$A_{D}$ is the air density and $E$ is the efficiency of the
turbine. ($0.5 \times V^{3}$) is velocity participating in a
calculation of Kinetic Energy to produce the mass of air on the
turbine \cite{arcGIS:2013}. Since Velocity is cubed in this
equation, doubling the wind speed would increase the power
output by eight. This is why a probability density graph is
better than just a yearly average. Also if the radius of the
turbine is doubled the power output would increase by four.
 \\
\indent The wind turbine that is under consideration for this
package has a radius of .127 meters. This gives a swept area of
approximately .051 m2. The maximum efficiency of any wind turbine
can only be $59\%$ but this wind turbine is closer to $40\%$
\cite{Windpower}\\

\indent With the power equation, turbine efficiency, and the
probability density graph, an approximation can be made about the
power output from the NOAA data. The annual output with this
turbine at this location was calculated to be around 37,000 Watt
hours. Without knowing exactly what instruments and draw that
the sensor package is going to have, there is no definite
answer that the purposed turbine will be able to power this
package by itself. 
